\begin{frame}{Avances}
	\begin{center}
	{\Large 	\textbf{Dimerización Calsecuestrina}}
	\end{center}
	\begin{columns}
		\begin{column}{0.5\textwidth}
			\begin{align*}
			\cee{R_0 + C & <=> R_1} \\
			\cee{R_1 + C & <=> R_2} \\
			& \vdots \\
			\cee{R_{n-1} + C & <=> R_n} \\
			\cee{2R_n & <=> R_D}
			\end{align*}
		\end{column}
		\begin{column}{0.5\textwidth}
			\begin{align*}
			R_i & = R_0 \binom{n}{i} \left(\frac{C}{K_C}\right)^i \\
			R_D & = \frac{R_0^2}{K_D} \left(\frac{C}{K_C}\right)^{2n}
			\end{align*}
			\begin{equation*}
			\sum_{i=0}^n R_i + 2 R_D = R_T
			\end{equation*}
		\end{column}
	\end{columns}
	\begin{align*}
	R & \rightarrow \text{Concentración de calsecuestrina} \\
	C & \rightarrow \text{Concentración de calcio} \\
	n & \rightarrow \text{Número de sitios de unión a calcio} \\
	K_C & \rightarrow \text{Constante de disociación $Ca^{2+}$-Calsecuestrina} \\
	K_D & \rightarrow \text{Constante de dimerización}
	\end{align*}
\end{frame}

\begin{frame}{Avances}
	
\end{frame}
